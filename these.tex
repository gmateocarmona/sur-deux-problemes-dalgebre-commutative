%Begin




\setcounter{page}{1}
%%%%%%%%%%%%%%%%%%%%%%%%%%%%%%%%%%%%%%%%%%%%%%%%%%%%%%%%%%%%%%%
\chapter*{INTRODUCTION}\thispagestyle{empty}
\addcontentsline{toc}{chapter}{Introduction}
\label{sec:intro}
\section*{}

Nous présentons ici les résultats que nous avons obtenus depuis 1968 sur les anneaux japonais, universellement japonais, excellents et les anneaux de \emph{Weierstrass}, et sur le problème des chaînes d'idéaux premiers dans les anneaux noethériens.

Nous avons classé ces résultats en deux parties : La première partie que nous avons intitulée \emph{``Sur la théorie des anneaux japonais et les questions qui s'y rattachent''} contient les résultats sur les anneaux japonais, universellement japonais et excellents, et les anneaux de \emph{Weierstrass}.

La deuxième partie contient les résultats sur les problèmes des chaînes d'idéaux premiers dans les anneaux noethériens.

Tous ces résultats sont donnés sans démonstrations, pour la seule raison que celles-ci ont été publiées dans nos articles parus, ou en cours de parution.

La plupart de ces résultats sont des généralisations de résultats dûs aux grands maîtres de ces vingt-cinq dernières années, principalement \emph{Chevalley}, \emph{Grothendieck}, \emph{Mori}, \emph{Nagata}, \emph{Samuel} et \emph{Zariski}, ou des réponses à certaines questions qu'ils ont laissées en suspens.

Nous tenons ici à remercier les Professeurs Alexander \emph{Grothendieck} et Pierre \emph{Samuel} qui ont dirigé nos travaux, le Professeur Henri \emph{Cartan} qui nous a fait l'honneur de présider ce jury et qui nous a proposé le second sujet de Thèse et les Professeurs Adrien \emph{Douady} et Jean-Luc \emph{Verdier} pour avoir accepté de faire partie du Jury.  

Nous remerciements vont également à toutes les secrétaires de la Faculté des Sciences de \emph{Dakar} qui ont assuré la dactylographie du manuscrit.

%%%%%%%%%%%%%%%%%%%%%%%%%%%%%%%%%%%%%%%%%%%%%%%%%%%%%%%%%%%%%%%
\chapter*{SUR LA THÉORIE DES ANNEAUX JAPONAIS ET LES QUESTIONS QUI S'Y RATTACHENT}\thispagestyle{empty}
\addcontentsline{toc}{chapter}{Sur la théorie des anneaux japonais et les questions qui s'y tarrachent}
\label{sec:a}
\section*{}

Cette première partie a trait au problème de la classification des anneaux locaux noethériens à partir des propriétés de leurs fibres formelles. Toutes ces propriétés ont été mises en évidence par le travail de géomètres japonais sur la finitude de la fermeture intégrale, principalement \emph{Mori} et \emph{Nagata}, et dégagées par \emph{Grothendieck}. Ces propriétés étalent en germes également dans les travaux de \emph{Chevalley} et \emph{Zariski} sur les complétés des anneaux locaux de la géométrie algébrique et dans les travaux de \emph{Samuel} sur des ``anneaux à noyau''. Cependant les résultats les plus décisifs dans cette théorie sont ceux de \emph{Nagata} sur les anneaux japonais, universellement japonais, les anneaux de \emph{Weierstrass} et les critiques jacobiennes, et de \emph{Grothendieck} sur les anneaux excellents.

Nos principales interventions dans cette théorie consistent essentiellement en des généralisations de certains résultats de ces auteurs et en l’établissement de certaines conjectures mises en évidence par les travaux.

%%%%%%%%%%%%%%%%%%%%%%%
\subsection*{I. Finitude de la fermeture intégrale}\label{sec:1}%
\addcontentsline{toc}{section}{I. Finitude de la fermeture intégrale}

{
Théorème 1. --- \it Soit $A$ un anneau semi-local noethérien. Alors, les conditions suivantes sont équivalentes~:
\begin{enumerate}
    \item[i)] Pour tout anneau quotient intègre $B$ de $A$, la clôture intégrale $\overline{B}$ de $B$ est un $B$-module de type fini.
    \item[ii)] Pour tout anneau quotient intègre $\overline{B}$ de $B$, la complété $\widehat{B}$ de $B$ est réduit.
\end{enumerate}
}
\vskip .3cm

% page 3 missing

conclusion du théorème 3, hypothèse qui est plus forte que la conjonction de nos hypothèses i) et ii).

Le raisonnement qui nous sert à prouver ce théorème permet de simplifier la démonstration de Zariski du théorème suivant~:
\vskip .3cm
{
Théorème 4 (Zariski). --- \it Soient $A$ un anneau semi-local noethérien intègre dont le séparé complété $\widehat{A}$ est normal et $B$ une $A$-algèbre finie intègre contenant $A$ et dont le corps des fractions est une extension séparable de celui de $A$. On suppose que, pour tout idéal premier $p$ de hauteur 1 de $B$, le séparé complété $\widehat{B}$ de $B$ est normal.
}
Le même raisonnement permet de simplifier et de généraliser sous la forme suivante, le théorème de ``pureté'' de \emph{Zariski-Nagata}.
\vskip .3cm
{
Théorème 5. --- \it Soient $S$ un schéma localement noethérien, $\psi: X \to S$ un morphisme fini et $x$ un point de $X$. On suppose vérifiées les conditions suivantes :
\begin{enumerate}
    \item[i)] $\mathcal{O}_{S, \psi(x)}$ est un anneau local régulier.
    \item[ii)] $x$ est un point unibranche de $X$ où $X$ vérifie ($S_1$).
    \item[iii)] $$\dim(\mathcal{O}_{X, x}) = \dim(\mathcal{O}_{S, \psi(x)})$$ (condition qui est satisfaite lorsque $X$ et $S$ sont intègres et $\psi$ dominante).
    \item[iv)] $\psi$ est non ramifié en toute généralisation $x'$ de $x$ dans $X$ telle que $$\dim(\mathcal{O}_{X, x'}) \leq 1$$.
\end{enumerate}
Alors $\psi$ est étale au point $x$, donc aussi dans un voisinage de $x$ dans $X$.
}
\vskip .3cm
{
Corollaire. --- \it Soient $A$ un anneau local régulier henselien et $B$ une $A$-algèbre intègre finie contenant $A$. On suppose que le morphisme canonique $\psi: \text{Spec}(B) \to \text{Spec}(A)$ est non ramifié en tout point $x$ de $\text{Spec}(B)$ de codimension $ \leq 1$ dans $\text{Spec}(B)$. Alors, $B$ est une $A$-algèbre étale.
}
\vskip .3cm
{
Théorème 6. --- \it Soient $A$ un anneau noethérien et $x$ un élément appartenant au radical de \emph{Jacobson} de $A$. On suppose vérifiées l'une des conditions suivantes~:
\begin{enumerate}
    \item[i)] Pour tout idéal maximal $m$ de $A$, l'anneau local $B = A_m$ est intègre et sa clôture intégrale $\overline{B}$ est un $B$-module de type fini et, pour tout idéal premier minimal $p'$ de $x\overline{B}$, $p' \bigcap A$ est un idéal premier de hauteur 1 (lorsque $A$ est un universellement caténaire, cette dernière condition est équivalente à $ht(x \overline{B}) = 1$ et, puisque $B$ est intègre, cela signifie que l'image de $x$ dans $\overline{B}$ est différente de 0).
    \item[ii)] Pour tout idéal maximal $m$ de $A$, l'anneau local $B = A_m$ est intègre et $B^{(1)}$ (notation de [4], 5.10.17]) est un $B$-module de type fini.
    \item[iii)] $A$ est caténaire et vérifié ($S_1$) et, pour tout idéal maximal $m$ de $A$, l'anneau local $B = A_m$ est équidimensionnel et, pour tout idéal premier minimal $q$ de $B = A_m$, posant $B_0 = B/q$, l'anneau $B_0^{(1)}$ est un $B_0$-module de type fini.
\end{enumerate}
On suppose, plus, vérifiées les deux conditions suivantes~:

%%%%%%%%%%%%%%%%%%%%%%%
\subsection*{II. Anneaux japonais}\label{sec:2}%
\addcontentsline{toc}{section}{II. Anneaux japonais}

%%%%%%%%%%%%%%%%%%%%%%%
\subsection*{III. Anneaux excellents}\label{sec:3}%
\addcontentsline{toc}{section}{III. Anneaux excellents}

%%%%%%%%%%%%%%%%%%%%%%%
\subsection*{IV. Anneaux excellents et critères jacobiens}\label{sec:4}%
\addcontentsline{toc}{section}{IV. Anneaux excellents et critères jacobiens}

%%%%%%%%%%%%%%%%%%%%%%%
\subsection*{V. Anneaux de Weierstrass}\label{sec:5}%
\addcontentsline{toc}{section}{V. Anneaux de Weierstrass}

%%%%%%%%%%%%%%%%%%%%%%%%%%%%%%%%%%%%%%%%%%%%%%%%%%%%%%%%%%%%%%%
\chapter*{ADDENDA}\thispagestyle{empty}
\addcontentsline{toc}{section}{Addenda}
\label{sec:add}
\section*{}

%%%%%%%%%%%%%%%%%%%%%%%%%%%%%%%%%%%%%%%%%%%%%%%%%%%%%%%%%%%%%%%
\chapter*{NOTE}\thispagestyle{empty}
\addcontentsline{toc}{section}{Note}
\label{sec:n1}
\section*{}

\subsection*{Chapitre I}

Les théorèmes de ce chapitre ont été annoncés au Colloque d’Algèbre tenu à \emph{Rennes} (France) du 19 au 22 janvier 1972. On trouvera des esquisses de démonstration dans notre article paru dans les Comptes Rendus du dit colloque. Pour ce qui est du théorème 6 (chap.I), nous en avions donné une démonstration dans notre article ``Un exemple d’anneau local noethérien japonais qui n’est pas formellement réduit'', C. Rend. Acad. Sci. Paris, t.274, p.1334--1337 (1972). Quant aux autres théorèmes du chapitre I, des démonstrations plus simples paraîtront très prochainement dans un note actuelle en préparation.

\subsection*{Chapitre II}

Le théorème 8 du chapitre II est démontré dans notre article : ``Sur la théorie des anneaux japonais'', C. Rend. Acad. Sci. Paris, t.271, p.73--75 (1970). Il généralise un résultat de \emph{Tate} (BGAC IV 23.1.3), ainsi que le théorème 13.

Le théorème 9 a été annoncé au Colloque d’Algèbre de \emph{Rennes} (loc.cit.); il en a été donné une démonstration dans notre article paru dans les Comptes Rendus de ce colloque.

Le théorème 10 a été démontré dans notre article : ``Sur la théorie des anneaux de \emph{Weierstrass}'', Bull. Sci. Math., t.95, 1971, p.223--225.

Le théorème 11 a été démontré dans notre article : ``Un exemple d’anneau local noethérien japonais qui n’est pas formellement réduit'', C. Rend. Acad. Sci. Paris, t.274, p.1334--1337 (1972).

Les théorèmes 12, 14, 15 ont été démontrés dans notre article : ``La réciproque d’un théorème de \emph{Kikuchi}'', J. of Math of Kyoto University, vol.11, n$^\circ$ 3 (1971), p.415--424. Leur démonstration s’appuie sur un résultat de \emph{Rees} (cf. son article : ``A note on analytically unramified local rings'' J. Math. Soc. (1961), p.24--28. Ils généralisent certains résultats de \emph{Kikuchi} (cf. son article ``On the finiteness of derived normal ringes of an affine ring'', J. Math. Soc. Japan, vol.15 n$^\circ$ 3, 1963, p.360--365) sur la finitude de la fermeture intégrale d’une algèbre affine.

\subsection*{Chapitre III}

Les théorèmes 17, 18 et 19 ont été démontrés dans deux articles : ``Sur la théorie des anneaux excellents en caractéristique $p$'' , Bull. Sci. Math., t.96, 1972, p.193--198 et ``Sur une note d’Ernst \emph{Kunz}, C.Rend.Acad.Sci.Paris, t.274, p.714--716 (1972). Ils généralisent des résultats de E. \emph{Kunz} (cf. son article : ``A characterisation of regular local ring of characteristic $p$'', Am. J. of Math 3 (1967) p.178--190).

Le théorème 20 a été démontré dans notre article : ``Sur la théorie des anneaux excellents en caractéristique $p$'', Bull. Sci. Math., t.96, 1972, p.193--198.

\subsection*{Chapitre IV}

Les théorèmes 21, 22, 23, 24, 25, 25 et 27 ont été énoncés dans notre note ``Un critère jacobien des points simples'' (à paraître aux C.R., Acad.Sci.Paris) sans démonstrations. Ces démonstrations paraîtront dans nos articles en préparation : ``Sur la théorie des anneaux excellents en caractéristique $p$, III'' et ``Sur la théorie des anneaux excellents en caractéristique $0$, I'' et ``Sur les algèbres jacobiennes''.

\subsection*{Chapitre V}

La démonstration du théorème 32 a été donnée dans notre article ``Sur la théorie des anneaux de \emph{Weierstrass}, I'' Bull. Soc. Math., t.95, 1971, p.223--225.

Quant aux autres théorèmes du chapitre, leurs démonstrations paraîtront dans nos articles : ``Sur la théorie des anneaux excellents en caractéristique zéro''.

%%%%%%%%%%%%%%%%%%%%%%%%%%%%%%%%%%%%%%%%%%%%%%%%%%%%%%%%%%%%%%%
\renewcommand{\bibname}{BIBLIOGRAPHIE}
\addcontentsline{toc}{chapter}{Bibliographie}
\def\refname{B\MakeLowercase{IBLIOGRAPHIE}}
\begin{thebibliography}{99}\thispagestyle{empty}

\bibitem{bass63}
  {\sc H. Bass} ---
  {\it On the ubiquity of Gorenstein ring}. Math. Zeit. 82 (1963), p. 8-68

\bibitem{grothendieckdieudonne}
  {\sc A. Grothendieck et J. Dieudonné} ---
  {\it Éléments de géométrie algébrique I}. Springer Verlag Berlin, Heidelberg, New-York

\end{thebibliography}



%%%%%%%%%%%%%%%%%%%%%%%%%%%%%%%%%%%%%%%%%%%%%%%%%%%%%%%%%%%%%%%
\chapter*{SUR LE PROBLÈME DE CHAÎNES D’IDÉAUX PREMIERS DANS LES ANNEAUX NOETHÉRIENS}\thispagestyle{empty}
\addcontentsline{toc}{chapter}{Sur le problème des chaînes d'idéaux premiers dans les anneaux noethériens}
\label{sec:b}
\section*{}




%%%%%%%%%%%%%%%%%%%%%%%%%%%%%%%%%%%%%%%%%%%%%%%%%%%%%%%%%%%%%%%
\chapter*{NOTE}\thispagestyle{empty}
\addcontentsline{toc}{section}{Note}
\label{sec:n2}
\section*{}





%End
