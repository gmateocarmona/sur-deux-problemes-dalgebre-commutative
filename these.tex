%Begin


\setcounter{page}{1}
%%%%%%%%%%%%%%%%%%%%%%%%%%%%%%%%%%%%%%%%%%%%%%%%%%%%%%%%%%%%%%%
\chapter*{INTRODUCTION}\thispagestyle{empty}
\addcontentsline{toc}{chapter}{Introduction}
\label{sec:intro}
\section*{}

Nous présentons ici les résultats que nous avons obtenus depuis 1968 sur les anneaux japonais, universellement japonais, excellents et les anneaux de \emph{Weierstrass}, et sur le problème des chaînes d'idéaux premiers dans les anneaux noethériens.

Nous avons classé ces résultats en deux parties : La première partie que nous avons intitulée \emph{``Sur la théorie des anneaux japonais et les questions qui s'y rattachent''} contient les résultats sur les anneaux japonais, universellement japonais et excellents, et les anneaux de \emph{Weierstrass}.

La deuxième partie contient les résultats sur les problèmes des chaînes d'idéaux premiers dans les anneaux noethériens.

Tous ces résultats sont donnés sans démonstrations, pour la seule raison que celles-ci ont été publiées dans nos articles parus, ou en cours de parution.

La plupart de ces résultats sont des généralisations de résultats dûs aux grands maîtres de ces vingt-cinq dernières années, principalement \emph{Chevalley}, \emph{Grothendieck}, \emph{Mori}, \emph{Nagata}, \emph{Samuel} et \emph{Zariski}, ou des réponses à certaines questions qu'ils ont laissées en suspens.

Nous tenons ici à remercier les Professeurs Alexander \emph{Grothendieck} et Pierre \emph{Samuel} qui ont dirigé nos travaux, le Professeur Henri \emph{Cartan} qui nous a fait l'honneur de présider ce jury et qui nous a proposé le second sujet de Thèse et les Professeurs Adrien \emph{Douady} et Jean-Luc \emph{Verdier} pour avoir accepté de faire partie du Jury.  

Nous remerciements vont également à toutes les secrétaires de la Faculté des Sciences de \emph{Dakar} qui ont assuré la dactylographie du manuscrit.

%%%%%%%%%%%%%%%%%%%%%%%%%%%%%%%%%%%%%%%%%%%%%%%%%%%%%%%%%%%%%%%
\chapter*{SUR LA THÉORIE DES ANNEAUX JAPONAIS ET LES QUESTIONS QUI S'Y RATTACHENT}\thispagestyle{empty}
\addcontentsline{toc}{chapter}{Sur la théorie des anneaux japonais et les questions qui s'y tarrachent}
\label{sec:a}
\section*{}

Cette première partie a trait au problème de la classification des anneaux locaux noethériens à partir des propriétés de leurs fibres formelles. Toutes ces propriétés ont été mises en évidence par le travail de géomètres japonais sur la finitude de la fermeture intégrale, principalement \emph{Mori} et \emph{Nagata}, et dégagées par \emph{Grothendieck}. Ces propriétés étalent en germes également dans les travaux de \emph{Chevalley} et \emph{Zariski} sur les complétés des anneaux locaux de la géométrie algébrique et dans les travaux de \emph{Samuel} sur des ``anneaux à noyau''. Cependant les résultats les plus décisifs dans cette théorie sont ceux de \emph{Nagata} sur les anneaux japonais, universellement japonais, les anneaux de \emph{Weierstrass} et les critiques jacobiennes, et de \emph{Grothendieck} sur les anneaux excellents.

Nos principales interventions dans cette théorie consistent essentiellement en des généralisations de certains résultats de ces auteurs et en l’établissement de certaines conjectures mises en évidence par les travaux.

%%%%%%%%%%%%%%%%%%%%%%%
\subsection*{I. Finitude de la fermeture intégrale}\label{sec:1}%
\addcontentsline{toc}{section}{I. Finitude de la fermeture intégrale}

{
Théorème 1. --- \it Soit $A$ un anneau semi-local noethérien. Alors, les conditions suivantes sont équivalentes~:
\begin{enumerate}
    \item[i)] Pour tout anneau quotient intègre $B$ de $A$, la clôture intégrale $\overline{B}$ de $B$ est un $B$-module de type fini.
    \item[ii)] Pour tout anneau quotient intègre $\overline{B}$ de $B$, la complété $\widehat{B}$ de $B$ est réduit.
\end{enumerate}
}
\vskip .3cm

% page 3 missing

conclusion du théorème 3, hypothèse qui est plus forte que la conjonction de nos hypothèses i) et ii).

Le raisonnement qui nous sert à prouver ce théorème permet de simplifier la démonstration de Zariski du théorème suivant~:
\vskip .3cm
{
Théorème 4 (Zariski). --- \it Soient $A$ un anneau semi-local noethérien intègre dont le séparé complété $\widehat{A}$ est normal et $B$ une $A$-algèbre finie intègre contenant $A$ et dont le corps des fractions est une extension séparable de celui de $A$. On suppose que, pour tout idéal premier $p$ de hauteur 1 de $B$, le séparé complété $\widehat{B}$ de $B$ est normal.
}
Le même raisonnement permet de simplifier et de généraliser sous la forme suivante, le théorème de ``pureté'' de \emph{Zariski-Nagata}.
\vskip .3cm
{
Théorème 5. --- \it Soient $S$ un schéma localement noethérien, $\psi: X \to S$ un morphisme fini et $x$ un point de $X$. On suppose vérifiées les conditions suivantes :
\begin{enumerate}
    \item[i)] $\mathcal{O}_{S, \psi(x)}$ est un anneau local régulier.
    \item[ii)] $x$ est un point unibranche de $X$ où $X$ vérifie ($S_1$).
    \item[iii)] $$\dim(\mathcal{O}_{X, x}) = \dim(\mathcal{O}_{S, \psi(x)})$$ (condition qui est satisfaite lorsque $X$ et $S$ sont intègres et $\psi$ dominante).
    \item[iv)] $\psi$ est non ramifié en toute généralisation $x'$ de $x$ dans $X$ telle que $$\dim(\mathcal{O}_{X, x'}) \leq 1$$.
\end{enumerate}
Alors $\psi$ est étale au point $x$, donc aussi dans un voisinage de $x$ dans $X$.
}
\vskip .3cm
{
Corollaire. --- \it Soient $A$ un anneau local régulier henselien et $B$ une $A$-algèbre intègre finie contenant $A$. On suppose que le morphisme canonique $\psi: \text{Spec}(B) \to \text{Spec}(A)$ est non ramifié en tout point $x$ de $\text{Spec}(B)$ de codimension $ \leq 1$ dans $\text{Spec}(B)$. Alors, $B$ est une $A$-algèbre étale.
}
\vskip .3cm
{
Théorème 6. --- \it Soient $A$ un anneau noethérien et $x$ un élément appartenant au radical de \emph{Jacobson} de $A$. On suppose vérifiées l'une des conditions suivantes~:
\begin{enumerate}
    \item[i)] Pour tout idéal maximal $m$ de $A$, l'anneau local $B = A_m$ est intègre et sa clôture intégrale $\overline{B}$ est un $B$-module de type fini et, pour tout idéal premier minimal $p'$ de $x\overline{B}$, $p' \bigcap A$ est un idéal premier de hauteur 1 (lorsque $A$ est un universellement caténaire, cette dernière condition est équivalente à $ht(x \overline{B}) = 1$ et, puisque $B$ est intègre, cela signifie que l'image de $x$ dans $\overline{B}$ est différente de 0).
    \item[ii)] Pour tout idéal maximal $m$ de $A$, l'anneau local $B = A_m$ est intègre et $B^{(1)}$ (notation de [4], 5.10.17]) est un $B$-module de type fini.
    \item[iii)] $A$ est caténaire et vérifié ($S_1$) et, pour tout idéal maximal $m$ de $A$, l'anneau local $B = A_m$ est équidimensionnel et, pour tout idéal premier minimal $q$ de $B = A_m$, posant $B_0 = B/q$, l'anneau $B_0^{(1)}$ est un $B_0$-module de type fini.
\end{enumerate}
On suppose, plus, vérifiées les deux conditions suivantes~:
\begin{itemize}
    \item[a)] $xA$ n'a qu'un seul idéal premier minimal $p$, $xAp = pAp$ et $\dim(Ap) = 1$.
    \item[b)] $A/p$ est intégralement clos.
\end{itemize}
Alors $A$ est intègre et intégralement clos, $p = xA$ et $\dim(A/xA) = \dim(A) - 1$.
}
\vskip .3cm
La démonstration du théorème 6 s'appuie sur le lemme suivant dû à l'auteur et à F. \emph{Ferrand}.
\vskip .3cm
{
Lemme. --- \it Soient $A$ un anneau noethérien et $x$ un élément de $A$ tel que $p = xA$ soit un idéal premier non minimal. Alors les conditions suivantes sont équivalentes :
\begin{itemize}
    \item[i)] $A$ est intègre.
    \item[ii)] $A$ est séparé pour la topologie $p$-adique.
\end{itemize}
}
\vskip .3cm
{
Corollaire. --- Soit \it $A$ un anneau noethérien. S'il existe un idéal premier de $A$ non minimal qui est monogène et contenu dans le radical de \emph{Jacobson} de $A$, alors $A$ est intègre.
}
\vskip .3cm
{\bf Remarque} : Le théorème 6 avec des hypothèses plus fortes est contenu dans la littérature sous le nom de ``lemme de \emph{Hironaka}''. En fait Hironaka l'a prouvé pour une $A$-algèbre intégrale essentiellement de type fini sur un corps, dans son article : ``A note on algebraic geometry over ground rings --- the invariance of Hilbert characteristic function under the specialization process'', Illinois Journal of Math. 2(1958) p. 355-356, \emph{Nagata} l'a établi sous l'hypothèse (i) en supposant de plus, $A$ intègre (cf. [10] 36.9 p. 134) et \emph{Grothendieck} ([4] 5.12.8) sous l'hypothèse (iii) en supposant de plus, $A$ local et réduit.
\vskip .3cm
{
Corollaire. --- \it Soient $A$ un anneau noethérien et $x$ un élément de $A$, tel que $p = xA$ soit un idéal premier non minimal de $A$ contenu dans le radical de \emph{Jacobson} de $A$ et que $A/p$ soit intégralement clos. Alors, $A$ est intègre et intégralement clos.
}
\vskip .3cm
{
Corollaire. --- \it Soient $A$ un anneau noethérien caténaire et $x_1,\cdots,x_n$ des éléments appartenant au radical de \emph{Jacobson} de $A$. On suppose que $\mathrm{ht} \left( \sum_{1 \leq i \leq n} x_i A \right) = n$ et que $a = \sum_{1 \leq i \leq n} x_i A$ n'a qu'un seul idéal premier minimal $p$, $pAp = aAp$ et $\mathrm{ht}(p) = n$

On suppose, de plus, satisfaites les conditions suivantes~:
\begin{itemize}
    \item[i)] $A$ satisfait à l'une des conditions (i),(ii), et (iii) du théorème 6 et tout anneau quotient intègre $B = A/q$ avec $q \subset p$ satisfait à l'une des conditions (i),(ii) et (iii) du théorème 6
    \item[ii)] $A/p$ est intégralement clos.
\end{itemize}
Alors $A$ est intègre et intégralement clos et $p = \sum_{1 \leq i \leq n} x_i A$.

En outre, pour tout entier $i \, , (1 \leq i \leq n)$, l'anneau quotient $A_i = A/ \sum_{1 \leq i \leq n} aj A$ est intègre et intégralement clos et $\dim(Ai) = \dim(A) - i$.
}
\vskip .3cm
\vskip .3cm
{
Corollaire. --- \it Soient $A$ un anneau noethérien et $x_1,\cdots,x_n$ des éléments appartenant au radical de \emph{Jacobson} de $A$. On suppose vérifiées les conditions suivantes~:
\begin{itemize}
    \item[i)] Pour tout point fermé $s$ de $S = \Spec(A)$ l'anneau local $\mathcal{O}_{S,s}$ est équidimensionnel.
    \item[ii)] $\mathrm{ht} \left( \sum_{1 \leq i \leq n} ai A \right) = a $, $a = \sum_{1 \leq i \leq n} ai A$ si $a$ n'a qu'un seul idéal premier minimal $p$, $pAp = aAp$ et $A/p$ est intégralement clos.
    \item[iii)] $A$ vérifie $(S_1)$.
\end{itemize}
On suppose de plus que l'une des conditions suivantes est satisfaite~:
\begin{itemize}
    \item[a)] $A$ est quotient d'un anneau de \emph{Cohen-Macualey}
    \item[b)] $A$ est universellement caténaire et, pour tout anneau quotient intègre, $B = A/q$ de $A$, avec $q \subset p$ et tout idéal maximal $m$ de $B$, la clôture intégrale de l'anneau local $B_m$ est un $B_m$-module de type fini.
\end{itemize}
Alors $A$ est intègre et intégralement clos et $p = \sum_{1 \leq i \leq n} xi A$.
}
\vskip .3cm
En outre, pour tout entier $i \, (1 \leq i \leq n)$, l'anneau quotient $A_i = A/ \sum_{1 \leq j \leq n} xjA$ est intègre et intégralement clos et $\dim(A_i) = \dim(A) - i$.
\vskip .3cm
{
Théorème 7. --- \it Soient $A$ un anneau noethérien intègre et $x$ un élément de $A$. On suppose vérifiées les conditions suivantes~:
\begin{itemize}
    \item[(i)] $A$ est séparé et complet pour la topologie $xA$-adique
    \item[(ii)] $xA$ n'a qu'un seul idéal premier minimal $p$ et, si $\overline{A}$ désigne la clôture intégrale de $A$, pour tout idéal premier minimal $p'$ de $x\overline{A}$, on a $p' \bigcap A = p$.
    \item[(iii)] La clôture intégrale de l'anneau quotient $A/p$ est un $A/p$-module de type fini.
\end{itemize}
Alors la clôture intégrale $\overline{A}$ de $A$ est un $A$-module de type fini.
}
\vskip .3cm
{
Corollaire. --- \it Soient $A$ un anneau noethérien et $x$ un élément de $A$. On suppose vérifiées les conditions suivantes~:
\begin{itemize}
    \item[(i)] $A$ est séparé et complet pour la topologie $xA$-adique.
    \item[(ii)] $xA$ n'a qu'un seul idéal premier minimal $p$.
    \item[(iii)] $A/p$ est intégralement clos et, pour tout idéal premier $p'$ de hauteur $1$ de la clôture intégrale de $A$ contenant $x$ on a $p' \bigcap A = p$.
\end{itemize}
Alors $A$ est intégralement clos et $p = xA$.
}

%%%%%%%%%%%%%%%%%%%%%%%
\subsection*{II. Anneaux japonais}\label{sec:2}%
\addcontentsline{toc}{section}{II. Anneaux japonais}

%%%%%%%%%%%%%%%%%%%%%%%
\subsection*{III. Anneaux excellents}\label{sec:3}%
\addcontentsline{toc}{section}{III. Anneaux excellents}

{
Définition. --- \it On dit qu’un anneau $A$ est excellent s’il est noethérien et s’il vérifie les conditions suivantes~:
\begin{itemize}
    \item[i)] $A$ est universellement caténaire.
    \item[ii)] Pour tout idéal premier $p$ de $A$, les fibres formelles de $Ap$ sont géométriquement régulières.
    \item[iii)] Pour tout quotient intègre $B$ de $A$ et toute extension radicielle $K'$ du corps des fractions $K$ de $B$, il existe une sous-$B$-algèbre finie $B'$ de $K'$, contenant $B$, ayant $K'$ pour corps des fractions et telle que l’ensemble des points réguliers de $\Spec(B')$ contienne un ouvert non vide de $\Spec(B')$.
\end{itemize}
}
\vskip .3cm
{
Théorème 17. --- \it Soient $A$ un anneau noethérien japonais contenant un corps de caractéristique $p \neq 0$, $K$ son corps des fractions. On suppose que $[K : K^p] < + \infty$. Alors $A$ vérifie les conditions (ii) et (iii) de la définition des anneaux excellents. En particulier, si $A$ est universellement caténaire, alors $A$ est excellent.
}
\vskip .3cm
{
Théorème 18. --- \it Soient un anneau noethérien contenant un corps de caractéristique $p \neq 0$, tel que $A^p \to A$ soit un homomorphisme fini. Alors $A$ satisfait les conditions (iii) et (iii) de la définition des anneaux excellents. 

En particulier, si $A$ est universellement caténaire, alors $A$ est excellent.
}
\vskip .3cm
{
Théorème 19. --- \it Soient $A$ un anneau noethérien contenant un corps de caractéristique $p \neq 0$ et $K$ son corps résiduel. On suppose que $$[K : K^p] < + \infty$$. Les assertions suivantes sont équivalentes~:
\begin{itemize}
    \item[(i)] $A^p \to A$ est un homomorphisme fini.
    \item[(ii)] Les fibres formelles de $A$ sont géométriquement régulières.
    \item[(iii)] Les fibres formelles de $A$ sont géométriquement régulières, autrement dit, $A$ est un anneau japonais.
    \item[(iv)] Les fibres formelles de $A$ aux points génériques des composantes irréductibles de $\Spec(A)$ sont géométriquement régulières.
\end{itemize}
}
\vskip .3cm
{
Théorème 20. --- \it Soient $A$ un anneau local noethérien contenant un corps de caractéristique $p \neq 0$, $K$ son corps résiduel et $I$ un idéal de $A$ (différent de $A$). On suppose vérifiées les conditions suivantes :
\begin{itemize}
    \item[(i)] $[K : K^p] < +\infty$
    \item[(ii)] $A$ est séparé et complet pour la topologie $I$-adique.
\end{itemize}
Alors, pour que les fibres formelles de $A/I$ soient géométriquement régulières, il faut et il suffit que les fibres formelles de $A$ soient géométriquement régulières.
}

%%%%%%%%%%%%%%%%%%%%%%%
\subsection*{IV. Anneaux excellents et critères jacobiens}\label{sec:4}%
\addcontentsline{toc}{section}{IV. Anneaux excellents et critères jacobiens}

%%%%%%%%%%%%%%%%%%%%%%%
\subsection*{V. Anneaux de Weierstrass}\label{sec:5}%
\addcontentsline{toc}{section}{V. Anneaux de Weierstrass}

{
Définition. --- \it On dit qu’un anneau $A$ est un anneau de \emph{Weierstrass} si $A$ vérifie les trois conditions suivantes~:
\begin{itemize}
    \item[(i)] $A$ est un anneau semi-local noethérien hensélien
    \item[(ii)] $A$ est un anneau universellement japonais
    \item[(iii)] Pour tout anneau quotient intègre $B$ de $A$, il existe un sous-anneau local régulier $B'$ de $B$, tel que $B$ soit un $B'$-module de type fini.
\end{itemize}
}
\vskip .3cm
Par exemple, tout anneau semi-local noethérien complet est un anneau de \emph{Weierstrass} excellent (\emph{Nagata}).
\vskip .3cm
{
Théorème 32. --- \it Soient $A$ un anneau de \emph{Weierstrass} local, $k$ son corps résiduel et $p$ l’exposant caractéristique de $A$. Si $\left[ k : k^p \right] < +\infty$, alors $A$ est un anneau excellent.
}
\vskip .3cm
Soit $K$ un corps valué. On dit que $K$ est quasicomplet si le complété $K'$ de $K$ est une extension séparable de $K$. Dans son article (6) L. \emph{Gerritzen} a montré que, pour que l’anneau des séries convergentes $K \left\{ \{ T \} \right\}$ à une variable sur un corps valué $K$ soit un anneau de \emph{Weierstrass}, il faut et il suffit que $K$ soit quasi-complet et, dans ce cas, tout anneau analytique sur $K$ est un anneau de \emph{Weierstrass}.

On en conclut donc que, si $K$ est un corps valué quasi-complet d’exposant caractéristique $p$, tel que $$\left[ K : K^p \right] < +\infty$$ alors tout anneau analytique sur $K$ est un anneau de \emph{Weierstrass} excellent, donc, compte tenu du théorème 22, on a le~:
\vskip .3cm
{
Théorème 33. --- \it Soient $K$ un corps valué quasi-complet d’exposant caractéristique $p$, tel que $$\left[ K : K^p \right] < +\infty$$, $A$ un anneau de séries convergentes sur $K$ et $$A' = A \left[ T_1, \ldots, T_r \right]$$ un anneau de polynômes à un nombre fini de variables sur $K$. Alors $A$ vérifie la condition $(J_k)$ du théorème 29.
}
\vskip .3cm
{
Théorème 34. --- \it Soient $K$ un corps valué parfait et $A$ et $A'$ deux anneaux analytiques sur $K$. On suppose que $A$ et $A'$ sont intègres et intégralement clos. Alors le produit tensoriel analytique de $A$ et $A'$ (cf. [10] 47 p. 199) est intègre et intégralement clos.
}
\vskip .3cm
{
Corollaire. --- \it Soient $X$ et $Y$ deux schémas affines de type fini sur un corps $k$. On suppose vérifiées les conditions suivantes~:
\begin{itemize}
    \item[(i)] Chacun des schémas $X$ et $Y$ possède un point rationnel sur $k$.
    \item[(ii)] $X$ et $Y$ sont normaux.
\end{itemize}
Alors $X \times_k Y$ est un schéma intègre et normal.
}
\vskip .3cm
{
Définition. --- \it Soit $A = K \left[ \left[ T_1, \ldots, T_n \right] \right]$ l’anneau des séries formelles à $n$ variables sur un corps $K$. On appelle série formelle algébrique à $n$ variables sur $K$, toute série formelle $f \in A$ qui est algébrique sur l’anneau des polynômes $K \quad T_1, \ldots, T_n $.

L’ensemble des séries formelles algébriques à $n$ variables sur $K$ est le hensélisé $K \quad T_1, \ldots, T_n $ de l'anneau local à l’origine de $X = \Spec \left( K \left[ T_1, \ldots, T_n \right] \right)$.
}
\vskip .3cm
{
Théorème 35. --- \it Soient $K$ un corps et $A = K \left[ T_1, \ldots, T_n \right]$ l’anneau des séries formelles algébriques sur $K$. Alors $A$ est un anneau de \emph{Weierstrass} local régulier et excellent de dimension $n$, dont le complété est l’anneau des séries formelles $K \left[ \left[ T_1, \ldots, T_n \right] \right]$.

En outre, pour tout idéal $a$ (non nécessairement premier), il existe un sous-anneau $B$ contenant $K$ de l’anneau quotient $B' = A/a$ qui est $K$-isomorphe à l'anneau des séries formelles algébriques à $s = \left( \dim B' \right)$ variables sur $K$, tel que $B'$ soit un $B$-module de type fini.
}
\vskip .3cm
{
Théorème 36. --- \it Soient $K$ un corps, $A$ et $B$ deux $K$-algèbres locales noethériennes qui sont $K$-isomorphes à des algèbres finies sur des anneaux de séries formelles algébriques sur $K$ et $\psi : A \to B$ un $K$-homomorphisme local. Alors, les deux conditions suivantes sont équivalentes~:
\begin{itemize}
    \item[(i)] $\psi$ fait de $B$ un $A$-module quasi fini.
    \item[(ii)] $\psi$ est un homomorphisme fini.
\end{itemize} 
\vskip .3cm
{
Théorème 37. --- \it Soit $K$ un corps. Alors tout anneau de séries formelles algébriques sur $K$ vérifie la condition $(J_k)$ du théorème 28.
}
\vskip .3cm

%%%%%%%%%%%%%%%%%%%%%%%%%%%%%%%%%%%%%%%%%%%%%%%%%%%%%%%%%%%%%%%
\chapter*{ADDENDA}\thispagestyle{empty}
\addcontentsline{toc}{section}{Addenda}
\label{sec:add}
\section*{}

\begin{itemize}
    \item[$1^\circ)$] Dans l’énoncé du théorème $2$, on peut remplacer la condition ``pour tout idéal premier'', par ``sauf pour un nombre fini d’idéaux premiers''.
    \item[$2^\circ)$] Les deux résultats suivants sont à rattacher à ceux du chapitre IV.
\end{itemize}
\vskip .3cm
{
Théorème $1'$. --- \it Soient $k$ un corps, $p$ son exposant caractéristique, $$A = k[X_1, \ldots, X_r]$$ un anneau de polynômes à un nombre fini de variables sur $A$ et $$B = A[[T_1, \ldots, T_s]]$$ un anneau de séries formelles à un nombre fini de variables sur $A$. Si $$[k : k^p] < +\infty$$, alors tout anneau de polynômes à un nombre fini de variables sur $B$ satisfait à la condition $(J_k)$.
}
\vskip .3cm
{
Théorème $2'$. --- \it Soient $k$ un corps, $p$ son exposant caractéristique, $$A = k[[X_1, \ldots, X_r]]$$ un anneau de séries formelles à un nombre fini de variables sur $k$ et $$B = A\{  X_1, \ldots, X_s \}$$ un anneau de séries restreintes sur $A$ (pour la topologie précédente). Si $$[k : k^p] < +\infty$$ alors tout anneau de polynômes à un nombre fini de variables sur B satisfait à la condition $(J_1)$.
}
\vskip .3cm
{
\textbf{Remarque}~: Le théorème $1'$, sous la condition : ``$k$ est de caractéristique zéro'', est connu par \emph{H. Matsumura} qui l’a utilisé pour prouver que le séparé complet pour une topologie linéaire d’une algèbre de type fini sur un corps de caractéristique $0$ est un anneau excellent.

%%%%%%%%%%%%%%%%%%%%%%%%%%%%%%%%%%%%%%%%%%%%%%%%%%%%%%%%%%%%%%%
\chapter*{NOTE}\thispagestyle{empty}
\addcontentsline{toc}{section}{Note}
\label{sec:n1}
\section*{}

\subsection*{Chapitre I}

Les théorèmes de ce chapitre ont été annoncés au Colloque d’Algèbre tenu à \emph{Rennes} (France) du 19 au 22 janvier 1972. On trouvera des esquisses de démonstration dans notre article paru dans les Comptes Rendus du dit colloque. Pour ce qui est du théorème 6 (chap.I), nous en avions donné une démonstration dans notre article ``Un exemple d’anneau local noethérien japonais qui n’est pas formellement réduit'', C. Rend. Acad. Sci. Paris, t.274, p.1334--1337 (1972). Quant aux autres théorèmes du chapitre I, des démonstrations plus simples paraîtront très prochainement dans un note actuelle en préparation.

\subsection*{Chapitre II}

Le théorème 8 du chapitre II est démontré dans notre article : ``Sur la théorie des anneaux japonais'', C. Rend. Acad. Sci. Paris, t.271, p.73--75 (1970). Il généralise un résultat de \emph{Tate} (BGAC IV 23.1.3), ainsi que le théorème 13.

Le théorème 9 a été annoncé au Colloque d’Algèbre de \emph{Rennes} (loc.cit.); il en a été donné une démonstration dans notre article paru dans les Comptes Rendus de ce colloque.

Le théorème 10 a été démontré dans notre article : ``Sur la théorie des anneaux de \emph{Weierstrass}'', Bull. Sci. Math., t.95, 1971, p.223--225.

Le théorème 11 a été démontré dans notre article : ``Un exemple d’anneau local noethérien japonais qui n’est pas formellement réduit'', C. Rend. Acad. Sci. Paris, t.274, p.1334--1337 (1972).

Les théorèmes 12, 14, 15 ont été démontrés dans notre article : ``La réciproque d’un théorème de \emph{Kikuchi}'', J. of Math of Kyoto University, vol.11, n$^\circ$ 3 (1971), p.415--424. Leur démonstration s’appuie sur un résultat de \emph{Rees} (cf. son article : ``A note on analytically unramified local rings'' J. Math. Soc. (1961), p.24--28. Ils généralisent certains résultats de \emph{Kikuchi} (cf. son article ``On the finiteness of derived normal ringes of an affine ring'', J. Math. Soc. Japan, vol.15 n$^\circ$ 3, 1963, p.360--365) sur la finitude de la fermeture intégrale d’une algèbre affine.

\subsection*{Chapitre III}

Les théorèmes 17, 18 et 19 ont été démontrés dans deux articles : ``Sur la théorie des anneaux excellents en caractéristique $p$'' , Bull. Sci. Math., t.96, 1972, p.193--198 et ``Sur une note d’Ernst \emph{Kunz}, C.Rend.Acad.Sci.Paris, t.274, p.714--716 (1972). Ils généralisent des résultats de E. \emph{Kunz} (cf. son article : ``A characterisation of regular local ring of characteristic $p$'', Am. J. of Math 3 (1967) p.178--190).

Le théorème 20 a été démontré dans notre article : ``Sur la théorie des anneaux excellents en caractéristique $p$'', Bull. Sci. Math., t.96, 1972, p.193--198.

\subsection*{Chapitre IV}

Les théorèmes 21, 22, 23, 24, 25, 25 et 27 ont été énoncés dans notre note ``Un critère jacobien des points simples'' (à paraître aux C.R., Acad.Sci.Paris) sans démonstrations. Ces démonstrations paraîtront dans nos articles en préparation : ``Sur la théorie des anneaux excellents en caractéristique $p$, III'' et ``Sur la théorie des anneaux excellents en caractéristique $0$, I'' et ``Sur les algèbres jacobiennes''.

\subsection*{Chapitre V}

La démonstration du théorème 32 a été donnée dans notre article ``Sur la théorie des anneaux de \emph{Weierstrass}, I'' Bull. Soc. Math., t.95, 1971, p.223--225.

Quant aux autres théorèmes du chapitre, leurs démonstrations paraîtront dans nos articles : ``Sur la théorie des anneaux excellents en caractéristique zéro''.

%%%%%%%%%%%%%%%%%%%%%%%%%%%%%%%%%%%%%%%%%%%%%%%%%%%%%%%%%%%%%%%
\renewcommand{\bibname}{BIBLIOGRAPHIE}
\addcontentsline{toc}{chapter}{Bibliographie}
\def\refname{B\MakeLowercase{IBLIOGRAPHIE}}
\begin{thebibliography}{99}\thispagestyle{empty}

\bibitem{bass63}
  {\sc H. Bass} ---
  {\it On the ubiquity of Gorenstein ring}. Math. Zeit. 82 (1963), p. 8-68

\bibitem{grothendieckdieudonne}
  {\sc A. Grothendieck et J. Dieudonné} ---
  {\it Éléments de géométrie algébrique I}. Springer Verlag Berlin, Heidelberg, New-York

\end{thebibliography}

%%%%%%%%%%%%%%%%%%%%%%%%%%%%%%%%%%%%%%%%%%%%%%%%%%%%%%%%%%%%%%%
\chapter*{SUR LE PROBLÈME DE CHAÎNES D’IDÉAUX PREMIERS DANS LES ANNEAUX NOETHÉRIENS}\thispagestyle{empty}
\addcontentsline{toc}{chapter}{Sur le problème des chaînes d'idéaux premiers dans les anneaux noethériens}
\label{sec:b}
\section*{}

%%%%%%%%%%%%%%%%%%%%%%%%%%%%%%%%%%%%%%%%%%%%%%%%%%%%%%%%%%%%%%%
\chapter*{NOTE}\thispagestyle{empty}
\addcontentsline{toc}{section}{Note}
\label{sec:n2}
\section*{}

Le théorème 1 apparaît pour la première fois, à notre connaissance, ici (Il semblerait cependant qu’il soit connu de \emph{Ratliff}). Sa démonstration se fait par récurrence sur $n = \dim(A)$ et s’appuie sur le fait suivant~: Si $x$ est un élément du radical de \emph{Jacobson} de $A$ n’appartenant à aucun idéal premier minimal de $A$, $\dim(A / xA) = \dim(A) - 1$

Le corollaire du théorème 1 découle du fameux théorème de \emph{Ratliff} qui dit que tout anneau local noethérien henselien caténaire est universellement caténaire (cf. notre article ``Anneaux hensenliens et conditions de chaînes, I'', Bull. Soc. Math.  France, t.98, 1970, p.9–31).

Le théorème 2 découle du théorème selon lequel tout anneau de séries formelles à un nombre fini de variables sur un anneau noethérien universellement caténaire est universellement caténaire (cf. notre article ``Anneaux hensenliens et conditions de chaînes, I'', Bull. Soc. Math. France, t.98, 1970, p.9–31).

Le théorème 3 est une réponse affirmative à une question de \emph{Grothendieck}. Il a été prouvé dans notre article ``Anneaux hensenliens et conditions de chaînes, IV'' (C. Rend. Acad. Sci. Paris, t.271, 1970, p.120–121).

Les théorèmes 4, 5 et 6 ont été éprouvés dans notre article ``Sur la théorie des anneaux excellents en caractéristique $p$'', (Bull. Soc. Math., t.96, 1972, p.193–198).

Le théorème 7 découle du théorème de normalité analytique (cf. notre exposé au Colloque d’Algèbre de \emph{Rennes}, 19–22 janvier 1972).

Le théorème 8 a été démontré dans notre article ``Anneaux hensenliens et conditions de chaînes, III''~: la formule des dimensions (C. Rend. Acad. Sci. Paris, t.270, 1970, p.696–698).

Les théorèmes 9, 10 et 12 découlent d’un théorème de Madame \emph{Flexor} (cf. notre article ``Anneaux hensenliens et conditions de chaînes, II'', la formule des dimensions, C. Rend. Acad. Sci. Paris, t.270, 1970, p.696–698).


%End
