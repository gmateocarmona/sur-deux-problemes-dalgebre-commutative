%Begin




\setcounter{page}{1}
%%%%%%%%%%%%%%%%%%%%%%%%%%%%%%%%%%%%%%%%%%%%%%%%%%%%%%%%%%%%%%%
\chapter*{INTRODUCTION}\thispagestyle{empty}
\addcontentsline{toc}{chapter}{Introduction}
\label{sec:intro}
\section*{}

Nous présentons ici les résultats que nous avons obtenus depuis 1968 sur les anneaux japonais, universellement japonais, excellents et les anneaux de \emph{Weierstrass}, et sur le problème des chaînes d'idéaux premiers dans les anneaux noethériens.

Nous avons classé ces résultats en deux parties : La première partie que nous avons intitulée \emph{``Sur la théorie des anneaux japonais et les questions qui s'y rattachent''} contient les résultats sur les anneaux japonais, universellement japonais et excellents, et les anneaux de \emph{Weierstrass}.

La deuxième partie contient les résultats sur les problèmes des chaînes d'idéaux premiers dans les anneaux noethériens.

Tous ces résultats sont donnés sans démonstrations, pour la seule raison que celles-ci ont été publiées dans nos articles parus, ou en cours de parution.

La plupart de ces résultats sont des généralisations de résultats dûs aux grands maîtres de ces vingt-cinq dernières années, principalement \emph{Chevalley}, \emph{Grothendieck}, \emph{Mori}, \emph{Nagata}, \emph{Samuel} et \emph{Zariski}, ou des réponses à certaines questions qu'ils ont laissées en suspens.

Nous tenons ici à remercier les Professeurs Alexander \emph{Grothendieck} et Pierre \emph{Samuel} qui ont dirigé nos travaux, le Professeur Henri \emph{Cartan} qui nous a fait l'honneur de présider ce jury et qui nous a proposé le second sujet de Thèse et les Professeurs Adrien \emph{Douady} et Jean-Luc \emph{Verdier} pour avoir accepté de faire partie du Jury.  

Nous remerciements vont également à toutes les secrétaires de la Faculté des Sciences de \emph{Dakar} qui ont assuré la dactylographie du manuscrit.

%%%%%%%%%%%%%%%%%%%%%%%%%%%%%%%%%%%%%%%%%%%%%%%%%%%%%%%%%%%%%%%
\chapter*{A. --- SUR LA THÉORIE DES ANNEAUX JAPONAIS ET LES QUESTIONS QUI S'Y RATTACHENT}\thispagestyle{empty}
\addcontentsline{toc}{chapter}{I. Sur la théorie des anneaux japonais et les questions qui s'y tarrachent}
\label{sec:a}
\section*{}

Cette première partie a trait

%%%%%%%%%%%%%%%%%%%%%%%
\subsection*{I. Finitude de la fermeture intégrale}\label{sec:1}%
\addcontentsline{toc}{section}{I. Finitude de la fermeture intégrale}

%%%%%%%%%%%%%%%%%%%%%%%
\subsection*{II. Anneaux japonais}\label{sec:2}%
\addcontentsline{toc}{section}{II. Anneaux japonais}

%%%%%%%%%%%%%%%%%%%%%%%
\subsection*{III. Anneaux excellents}\label{sec:3}%
\addcontentsline{toc}{section}{III. Anneaux excellents}

%%%%%%%%%%%%%%%%%%%%%%%
\subsection*{IV. Anneaux excellents et critères jacobiens}\label{sec:4}%
\addcontentsline{toc}{section}{IV. Anneaux excellents et critères jacobiens}

%%%%%%%%%%%%%%%%%%%%%%%
\subsection*{V. Anneaux de Weierstrass}\label{sec:5}%
\addcontentsline{toc}{section}{V. Anneaux de Weierstrass}

%%%%%%%%%%%%%%%%%%%%%%%%%%%%%%%%%%%%%%%%%%%%%%%%%%%%%%%%%%%%%%%
\chapter*{ADDENDA}\thispagestyle{empty}
\addcontentsline{toc}{chapter}{Addenda}
\label{sec:a}
\section*{}

%%%%%%%%%%%%%%%%%%%%%%%%%%%%%%%%%%%%%%%%%%%%%%%%%%%%%%%%%%%%%%%
\chapter*{NOTE}\thispagestyle{empty}
\addcontentsline{toc}{chapter}{Note}
\label{sec:a}
\section*{}

\subsection*{Chapitre I}

\subsection*{Chapitre II}

\subsection*{Chapitre III}

\subsection*{Chapitre IV}

\subsection*{Chapitre V}


%%%%%%%%%%%%%%%%%%%%%%%%%%%%%%%%%%%%%%%%%%%%%%%%%%%%%%%%%%%%%%%
\renewcommand{\bibname}{BIBLIOGRAPHIE}
\addcontentsline{toc}{chapter}{Bibliographie}
\def\refname{B\MakeLowercase{IBLIOGRAPHIE}}
\begin{thebibliography}{99}\thispagestyle{empty}

\bibitem{bass63}
  {\sc H. Bass} ---
  {\it On the ubiquity of Gorenstein ring}. Math. Zeit. 82 (1963), p. 8-68

\bibitem{grothendieckdieudonne}
  {\sc A. Grothendieck et J. Dieudonné} ---
  {\it Éléments de géométrie algébrique I}. Springer Verlag Berlin, Heidelberg, New-York

\end{thebibliography}



%%%%%%%%%%%%%%%%%%%%%%%%%%%%%%%%%%%%%%%%%%%%%%%%%%%%%%%%%%%%%%%
\chapter*{SUR LE PROBLÈME DE CHAÎNES D’IDÉAUX PREMIERS DANS LES ANNEAUX NOETHÉRIENS}\thispagestyle{empty}
\addcontentsline{toc}{chapter}{Sur le problème des chaînes d'idéaux premiers dans les anneaux noethériens}
\label{sec:a}
\section*{}




%%%%%%%%%%%%%%%%%%%%%%%%%%%%%%%%%%%%%%%%%%%%%%%%%%%%%%%%%%%%%%%
\chapter*{NOTE}\thispagestyle{empty}
\addcontentsline{toc}{chapter}{Note}
\label{sec:a}
\section*{}





%End
