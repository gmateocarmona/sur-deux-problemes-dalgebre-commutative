\documentclass[12pt,twoside]{report} %openright
	%\usepackage[utf8]{inputenc}
 	\usepackage[greek,english]{babel}
    \usepackage{yfonts}
	\usepackage{latexsym}
	\usepackage{amsmath}
	\usepackage{amssymb}
	\usepackage{amsthm}
	\usepackage{stmaryrd}
	\usepackage{pb-diagram}
	\usepackage{amscd}
	\usepackage{mathrsfs}
	\usepackage{enumerate}
	\usepackage{color}
	\usepackage{cite}
	%\usepackage{graphicx}
	\usepackage[all]{xy}
	
	 
	\usepackage{graphicx}
\graphicspath{ {./images/} }
	
	%\usepackage{fixltx2e}
	\usepackage{extarrows}
	\usepackage{datetime}
	\usepackage{mathtools}
    \usepackage{remreset}
    \usepackage{afterpage}
    \usepackage{verbatim}

    \usepackage[svgnames]{xcolor}
	\usepackage{hyperref}
\hypersetup{
    colorlinks=true,
    linkcolor=[RGB]{61,82,62},
    filecolor=[RGB]{61,82,62},
    urlcolor=[RGB]{61,82,62},
    citecolor=[RGB]{61,82,62},
    pdftitle={Seydi, Hamet. Sur deux problèmes d'Algèbre Commutative. Thèse. Sc. math. Paris XI-Orsay. 1973. N$^\circ$ 1178. Transcription by M. Carmona et al. Draft},
    pdfauthor={Alexander Grothendieck},
    pdflang={fr}
}

\usepackage{bookmark}
    %\usepackage{csquotes}
    \usepackage{etoc}

\usepackage{pdflscape}
	
	\usepackage{fontspec}
	\setmainfont{EB Garamond}
	\usepackage{float}
	\usepackage[12pt]{moresize}
	\usepackage{titlesec}
	\usepackage[twoside,top=1.5in,bottom=1.5in,left=1in,right=1.3in,headheight=16pt,headsep=30pt,footskip=40pt]{geometry}

	\linespread{1.25}
%	\linespread{1.05}


%\NeedsTeXFormat{LaTeX2e}
%\ProvidesPackage{quiver}[2021/01/11 quiver]
\RequirePackage{tikz-cd}
\RequirePackage{amssymb}
\usetikzlibrary{calc}
\usetikzlibrary{decorations.pathmorphing}


% A TikZ style for curved arrows of a fixed height, due to AndréC.
\tikzset{curve/.style={settings={#1},to path={(\tikztostart)
    .. controls ($(\tikztostart)!\pv{pos}!(\tikztotarget)!\pv{height}!270:(\tikztotarget)$)
    and ($(\tikztostart)!1-\pv{pos}!(\tikztotarget)!\pv{height}!270:(\tikztotarget)$)
    .. (\tikztotarget)\tikztonodes}},
    settings/.code={\tikzset{quiver/.cd,#1}
        \def\pv##1{\pgfkeysvalueof{/tikz/quiver/##1}}},
    quiver/.cd,pos/.initial=0.35,height/.initial=0}

% TikZ arrowhead/tail styles.
\tikzset{tail reversed/.code={\pgfsetarrowsstart{tikzcd to}}}
\tikzset{2tail/.code={\pgfsetarrowsstart{Implies[reversed]}}}
\tikzset{2tail reversed/.code={\pgfsetarrowsstart{Implies}}}
% TikZ arrow styles.
\tikzset{no body/.style={/tikz/dash pattern=on 0 off 1mm}}

	% use osf for figures in math mode
	\DeclareSymbolFont{digits}{\encodingdefault}{\rmdefault}{m}{n}
	\SetSymbolFont{digits}{normal}{\encodingdefault}{\rmdefault}{m}{n}
	\DeclareMathSymbol{0}{\mathalpha}{digits}{"30}
	\DeclareMathSymbol{1}{\mathalpha}{digits}{"31}
	\DeclareMathSymbol{2}{\mathalpha}{digits}{"32}
	\DeclareMathSymbol{3}{\mathalpha}{digits}{"33}
	\DeclareMathSymbol{4}{\mathalpha}{digits}{"34}
	\DeclareMathSymbol{5}{\mathalpha}{digits}{"35}
	\DeclareMathSymbol{6}{\mathalpha}{digits}{"36}
	\DeclareMathSymbol{7}{\mathalpha}{digits}{"37}
	\DeclareMathSymbol{8}{\mathalpha}{digits}{"38}
	\DeclareMathSymbol{9}{\mathalpha}{digits}{"39}
	% helvetica for sans in math
	\DeclareMathAlphabet{\mathsf}{\encodingdefault}{phv}{m}{n}
	%\usepackage{showkeys}
	%\usepackage{showframe}
	\usepackage{fancyhdr}
	%\fancyhead[CO]{\rightmark}
	\fancyhead[CO]{}
	%\fancyhead[CE]{\leftmark}
	\fancyhead[CE]{}
	\fancyhead[RE]{}
	\fancyhead[LO]{}
	\fancyhead[RO]{}
	\fancyhead[LE]{}
	\cfoot{\thepage}
	\renewcommand{\headrulewidth}{0.0pt}
	\titleformat\section{\normalfont\Large\centerline{\hss\rule{1.3in}{.01in}\hss}}{\thesection}{1em}{}
    \titleformat{\chapter}[display]{}{\huge\filcenter { \thechapter}}{12pt}{\large \filcenter}

%\newcommand\hmmax{0}
%\newcommand\bmmax{0}

%%%%%%%%%%%%%%%%%%%%%%%%%%%%%%%%%%%%%%%%%%%%%%

	% arrows
	\newcommand{\xto}{\xrightarrow}
	\renewcommand{\to}{\longrightarrow}
	\def\rinto{\HorizontalMap\rthooka-\empty-\rhla}
	\def\linto{\HorizontalMap\lhla-\empty-\lthooka}
	\def\dinto{\VerticalMap\dthookb|\empty|\dhla}
	\def\uinto{\VerticalMap\uhla|\empty|\uthookb}

	% categories
	\def\cC{{\cal C}}
	\def\cF{{\cal F}}
	\def\cG{{\cal G}}
	\def\cH{{\cal H}}
	\def\cU{{\cal U}}

	% operators
	\DeclareMathOperator{\Hom}{Hom}
	\DeclareMathOperator{\Aut}{Aut}
	\DeclareMathOperator{\Is}{Is}

	% symbols
	\renewcommand{\emptyset}{\varnothing}
	\renewcommand{\phi}{\varphi}
	\newcommand{\und}{\underline}

	% environments
	\theoremstyle{plain}
		\newtheorem{lemma}{Lemma}[section]
		\newtheorem{corollary}[lemma]{Corollary}
		\newtheorem{proposition}[lemma]{Proposition}
		\newtheorem{theorem}[lemma]{Theorem}
		\newtheorem*{theorem*}{Theorem}
		\newtheorem*{corollary*}{Corollary}
		\newtheorem{conjecture}{Conjecture}
	\theoremstyle{definition}
		\newtheorem{definition}[lemma]{Definition}
		\newtheorem{example}[lemma]{Example}
		\newtheorem{remark}[lemma]{Remark}
		\newtheorem{construction}[lemma]{Construction}

%%%%%%%%%%%%%%%%%%%%%%%%%%%%%%%%%%%%%%%%%%%%%%

\makeatletter
\newcommand*{\toccontents}{\@starttoc{toc}}
\makeatother

\makeatletter
\@addtoreset{footnote}{section}
\makeatother

%\usepackage{sectsty}
%\partfont{\LARGE}

\begin{document}
\setcounter{tocdepth}{2}
\sloppy

\newgeometry{left=1.5in,right=1.5in, bottom=1.5in}

\thispagestyle{empty}
\vspace*{\fill}
\begin{center}
    {\Huge Sur deux problèmes} \\
    \vspace{0.5em}
    {\Huge d'Algèbre Commutative}
\end{center}
\vspace*{\fill}
\noindent\makebox[\linewidth]{\rule{1.3in}{0.01in}}
\vspace*{\fill}
\begin{center}
    {\Large par} \\
    \vspace{0.5em}
    {\huge Hamet Seydi}
\end{center}
\vspace*{\fill}
% \begin{center}
%    {\Large Transcription by} \\
%    \vspace{1.1em}
%    \includegraphics[width=70mm]{logo.pdf}
%\end{center}
%\vspace*{\fill}


%\begin{tikzpicture}[remember picture,overlay]
%    \draw[line width=10pt,color=DarkKhaki]
%        ([shift={(-0.5\pgflinewidth,-0.5\pgflinewidth)}]current page.north west)
%        rectangle
%        ([shift={(0.5\pgflinewidth,0.5\pgflinewidth)}]current page.south east);
%\end{tikzpicture}

\newpage

\thispagestyle{empty}
\vspace*{\fill}
\noindent {\large ORSAY} \\
\vspace{0.5em}
Série A \\
\vspace{0.5em}
{\bf N$^\circ$ d'ordre : 1178}
\begin{center}
    {\large \bf THÈSES} \\
    \vspace{0.5em}
    présentées \\ 
    à la \\ 
    FACULTÉ DES SCIENCES D'ORSAY \\ 
    pour obtenir \\ 
    le grade de Docteur ès-Sciences Mathématiques \\ 
    par \\ 
    Hamet SEYDI
\end{center}
\vspace*{\fill}
\begin{center}
    1ère Thèse : SUR DEUX PROBLÈMES D'ALGÈBRE COMMUTATIVE \\
    \vspace{0.5em}
    2ème Thèse : PROPOSITIONS DONNÉES PAR LA FACULTÉ
\end{center}
\vspace*{\fill}
\begin{center}
    Soutenues le 15 Octobre 1973, devant la Commission d'examen \\
    \vspace{1em}
    MM. CARTAN \quad\quad Président \\ 
    \vspace{0.5em}
\begin{tabular}{@{}l@{\hspace{1.5cm}}r@{}}
SAMUEL &  \\
VERDIER & Examinateurs\\
DOUADY & \\
\end{tabular}
\end{center}
\vspace*{\fill}

\newpage 
\thispagestyle{empty}

\begin{center}
Sur deux problèmes d'Algèbre Commutative, Hamet Seydi
\end{center}

\vspace*{\fill}

This transcription is derived from an unpublished scan. This project was carried out by researchers and volunteers under the supervision of Mateo Carmona.

\bigskip

How to cite:

Seydi, Hamet. \emph{Sur deux problèmes d'Algèbre Commutative}. Thèse. Sc. math. Paris XI-Orsay. 1973. N$^\circ$ 1178. Transcription by M. Carmona et al. Draft, \monthname[\the\month] \the\year.

\newpage
\thispagestyle{empty}
\mbox{}

%\pagestyle{fancyplain}

\renewcommand{\contentsname}{SOMMAIRE}
\tableofcontents\thispagestyle{empty}

%%%%%%%%%%%%%%%%%%%%%%%%%%%%%%%%%%%%%%%%%

%Begin




\setcounter{page}{1}
%%%%%%%%%%%%%%%%%%%%%%%%%%%%%%%%%%%%%%%%%%%%%%%%%%%%%%%%%%%%%%%
\chapter*{INTRODUCTION}\thispagestyle{empty}
\addcontentsline{toc}{chapter}{Introduction}
\label{sec:intro}
\section*{}

Nous présentons ici les résultats que nous avons obtenus depuis 1968 sur les anneaux japonais, universellement japonais, excellents et les anneaux de \emph{Weierstrass}, et sur le problème des chaînes d'idéaux premiers dans les anneaux noethériens.

Nous avons classé ces résultats en deux parties : La première partie que nous avons intitulée \emph{``Sur la théorie des anneaux japonais et les questions qui s'y rattachent''} contient les résultats sur les anneaux japonais, universellement japonais et excellents, et les anneaux de \emph{Weierstrass}.

La deuxième partie contient les résultats sur les problèmes des chaînes d'idéaux premiers dans les anneaux noethériens.

Tous ces résultats sont donnés sans démonstrations, pour la seule raison que celles-ci ont été publiées dans nos articles parus, ou en cours de parution.

La plupart de ces résultats sont des généralisations de résultats dûs aux grands maîtres de ces vingt-cinq dernières années, principalement \emph{Chevalley}, \emph{Grothendieck}, \emph{Mori}, \emph{Nagata}, \emph{Samuel} et \emph{Zariski}, ou des réponses à certaines questions qu'ils ont laissées en suspens.

Nous tenons ici à remercier les Professeurs Alexander \emph{Grothendieck} et Pierre \emph{Samuel} qui ont dirigé nos travaux, le Professeur Henri \emph{Cartan} qui nous a fait l'honneur de présider ce jury et qui nous a proposé le second sujet de Thèse et les Professeurs Adrien \emph{Douady} et Jean-Luc \emph{Verdier} pour avoir accepté de faire partie du Jury.  

Nous remerciements vont également à toutes les secrétaires de la Faculté des Sciences de \emph{Dakar} qui ont assuré la dactylographie du manuscrit.

%%%%%%%%%%%%%%%%%%%%%%%%%%%%%%%%%%%%%%%%%%%%%%%%%%%%%%%%%%%%%%%
\chapter*{A. --- SUR LA THÉORIE DES ANNEAUX JAPONAIS ET LES QUESTIONS QUI S'Y RATTACHENT}\thispagestyle{empty}
\addcontentsline{toc}{chapter}{I. Sur la théorie des anneaux japonais et les questions qui s'y tarrachent}
\label{sec:a}
\section*{}

Cette première partie a trait

%%%%%%%%%%%%%%%%%%%%%%%
\subsection*{I. Finitude de la fermeture intégrale}\label{sec:1}%
\addcontentsline{toc}{section}{I. Finitude de la fermeture intégrale}

%%%%%%%%%%%%%%%%%%%%%%%
\subsection*{II. Anneaux japonais}\label{sec:2}%
\addcontentsline{toc}{section}{II. Anneaux japonais}

%%%%%%%%%%%%%%%%%%%%%%%
\subsection*{III. Anneaux excellents}\label{sec:3}%
\addcontentsline{toc}{section}{III. Anneaux excellents}

%%%%%%%%%%%%%%%%%%%%%%%
\subsection*{IV. Anneaux excellents et critères jacobiens}\label{sec:4}%
\addcontentsline{toc}{section}{IV. Anneaux excellents et critères jacobiens}

%%%%%%%%%%%%%%%%%%%%%%%
\subsection*{V. Anneaux de Weierstrass}\label{sec:5}%
\addcontentsline{toc}{section}{V. Anneaux de Weierstrass}

%%%%%%%%%%%%%%%%%%%%%%%%%%%%%%%%%%%%%%%%%%%%%%%%%%%%%%%%%%%%%%%
\chapter*{ADDENDA}\thispagestyle{empty}
\addcontentsline{toc}{chapter}{Addenda}
\label{sec:a}
\section*{}

%%%%%%%%%%%%%%%%%%%%%%%%%%%%%%%%%%%%%%%%%%%%%%%%%%%%%%%%%%%%%%%
\chapter*{NOTE}\thispagestyle{empty}
\addcontentsline{toc}{chapter}{Note}
\label{sec:a}
\section*{}

\subsection*{Chapitre I}

\subsection*{Chapitre II}

\subsection*{Chapitre III}

\subsection*{Chapitre IV}

\subsection*{Chapitre V}


%%%%%%%%%%%%%%%%%%%%%%%%%%%%%%%%%%%%%%%%%%%%%%%%%%%%%%%%%%%%%%%
\renewcommand{\bibname}{BIBLIOGRAPHIE}
\addcontentsline{toc}{chapter}{Bibliographie}
\def\refname{B\MakeLowercase{IBLIOGRAPHIE}}
\begin{thebibliography}{99}\thispagestyle{empty}

\bibitem{bass63}
  {\sc H. Bass} ---
  {\it On the ubiquity of Gorenstein ring}. Math. Zeit. 82 (1963), p. 8-68

\bibitem{grothendieckdieudonne}
  {\sc A. Grothendieck et J. Dieudonné} ---
  {\it Éléments de géométrie algébrique I}. Springer Verlag Berlin, Heidelberg, New-York

\end{thebibliography}



%%%%%%%%%%%%%%%%%%%%%%%%%%%%%%%%%%%%%%%%%%%%%%%%%%%%%%%%%%%%%%%
\chapter*{SUR LE PROBLÈME DE CHAÎNES D’IDÉAUX PREMIERS DANS LES ANNEAUX NOETHÉRIENS}\thispagestyle{empty}
\addcontentsline{toc}{chapter}{Sur le problème des chaînes d'idéaux premiers dans les anneaux noethériens}
\label{sec:a}
\section*{}




%%%%%%%%%%%%%%%%%%%%%%%%%%%%%%%%%%%%%%%%%%%%%%%%%%%%%%%%%%%%%%%
\chapter*{NOTE}\thispagestyle{empty}
\addcontentsline{toc}{chapter}{Note}
\label{sec:a}
\section*{}





%End


%%%%%%%%%%%%%%%%%%%%%%%%%%%%%%%%%%%%%%%%%

\newpage
%\thispagestyle{empty}
\mbox{}
\thispagestyle{empty}
%\begin{tikzpicture}[remember picture,overlay]
%    \draw[line width=10pt,color=DarkKhaki]
%        ([shift={(-0.5\pgflinewidth,-0.5\pgflinewidth)}]current page.north west)
%        rectangle
%        ([shift={(0.5\pgflinewidth,0.5\pgflinewidth)}]current page.south east);
%\end{tikzpicture}

\end{document}



%End
